\documentclass[a4paper, 12pt]{article}
\usepackage[T1]{fontenc}
\usepackage[scale=1,angle=0,opacity=1,color=black!60]{background}
\usepackage{tikzpagenodes}
\usepackage{lastpage}
\usepackage{lmodern}
\usepackage{float}
\usepackage[textwidth=420pt,textheight=630pt]{geometry}
\setlength{\oddsidemargin}{15.5pt}
%\usepackage[none]{hyphenat} %no cortar palabras

\usepackage[spanish, activeacute]{babel} %Definir idioma español
\usepackage[utf8]{inputenc} %Codificacion utf-8
\backgroundsetup{contents={}} %Saca el 'draft'
\definecolor{mygray}{rgb}{0.9,0.9,0.9}

\usepackage{listings}
\lstset{
    basicstyle=\footnotesize,
    backgroundcolor=\color{mygray},         
    breaklines=true,
    breakatwhitespace=true,   
    postbreak=\mbox{\textcolor{red}{$\hookrightarrow$}\space},              
    captionpos=b,                    
    keepspaces=true,                 
    numbers=left,                    
    numbersep=5pt,                  
    showspaces=false,                
    showstringspaces=false,
    showtabs=false,
    tabsize=4,
    language=C,
    frame=none,
    title=\lstname
}

\def\labelitemi{$\bullet$}

\begin{document}		
	% TÍTULO, AUTORES Y FECHA
	\begin{titlepage}
		\vspace*{\fill}
		\begin{center}
			\Large 75.08 Sistemas Operativos \\
			\Huge Lab Shell - Parte 1 \\
			\bigskip\bigskip\bigskip
			\large\textbf{Nombre y Apellido:} Damián Cassinotti \\
			\textbf{Padrón:} 96618 \\
			\textbf{Fecha de Entrega:} 20/04/2018\\
					
		\end{center}
		\vspace*{\fill}
	\end{titlepage}
	\pagenumbering{arabic}
	\newpage
			
	% ÍNDICE
	\tableofcontents
	\newpage
	%\pagenumbering{arabic}
	
	\section{Invocación de comandos}
		El Lab de Shell consiste en implementar el comportamiento básico de un shell teniendo ya un esqueleto predefinido. El comportamiento pedido para esta entrega será el encargado de ejecutar comando básicos utilizando parámetros y variables de entorno.\\
		
		En cada caso veremos una breve explicación de lo pedido junto con el código de la solución. No mostraremos el código del shell completo, sino en cada parte nos centraremos en el necesario para cumplir con lo pedido. Se aclara que los ejercicios son iterativos e incrementales. Es decir, en cada ejercicio se suponen hechos los anteriores, contando con el código necesario.\\
		
		Al finalizar las explicaciones de cada punto se incluirá el código completo de los archivos modificados, sin incluir el esqueleto entero.
		\subsection{Búsqueda en \$PATH}
		El primer objetivo del lab consiste en poder ejecutar programas que se encuentren en el path de nuestra computadora. Esto es, poder ejecutarlos solamente llamandolos por su nombre y no por su ruta completa. Lo que hace el sistema es buscar en cada una de los directorios ubicados en la variable de entorno PATH para luego ejecutarlo.\\
		
		Para esto se utilizó la funcion \textit{execvpe}, de la familia de funciones \textit{exec}. De toda la familia de funciones se eligió \textit{execvpe} por varios motivos. Primero, para cumplir con los buscado en este punto, que es poder indicarle al sistema que busque el archivo deseado en el path. A diferencia de otras funciones de la familia, las funciones que en su numbre contienen la letra \textbf{p} son aquellas que reciben el nombre del archivo para buscar en el path. También pueden recibir la ruta absoluta.\\
		
		Los otros dos motivos por los cuales se eligió esta función es por la posibilidad de poder pasarle argumentos y variables de entorno externas. Esto se analizará en profundidad más adelante.\\
		
		A continuación se presenta el código que se agregó o modificó del esqueleto del shell. Recordar que en caso de archivos ya existentes, se mostrará solamente la sección modificada.\\
		\lstinputlisting{exec.h}
		\bigskip\bigskip\bigskip
		\lstinputlisting[linerange={53-55,62-70,101-102}, numbers=none]{exec.c}
		
		\textbf{¿Cuáles son las diferencias entre la syscall execve(2) y la familia de wrappers proporcionados por la librería estándar de C (libc) exec(3)?} Como lo indica la pregunta, la principal diferencia entre execve(2) y exec(3) es su naturaleza: el primero es una syscall, mientras que el segundo es una familia de funciones. Las funciones de la familia funcionan como wrappers de execve, es decir, internamente ejecutan la syscall. Este empaquetado permite la ejecución de binarios que se encuentren en el PATH.
		
		\subsection{Argumentos del programa}
		En este caso, el objetivo es que los ejecutables puedan recibir argumentos. Esto se logra pasandole los argumentos deseados a la función \textit{execvpe} utilizada en el punto anterior. Este fue uno de los elementos por los cuales se decidió usar dicha función, ya que permite el pasaje de parámetros.\\
		
		De esta forma, la función cumple con dos de los objetivos pedidos. No se mostrará código en esa sección ya que el implementado en el punto anterior alcanza para resolver rambos problemas.\\
	
		\subsection{Expansión de variables de entorno}
		El último de los objetivos de este lab es realizar la expansión de variables de entorno. Para esto se necesita poder reconocerlas y encontrar su verdadero valor.\\
		
		Las variables de entorno a reemplazar son indicadas mediante el caracter \$. De esta forma, lo que necesitamos es buscar cuáles de los argumentos de nuestro programa comienza con el caracter \$ para luego reemplazazrlos por su verdadero valor. Para obtener el valor verdadero se utilizará la función \textit{getenv(3)}, la cual dado el nombre de una variable de entorno devuelve su valor. Una vez obtenido dicho valor, se reemplazará en los argumentos del programa.\\
		 
		Nuestor esqueleto ya cuenta con una función dedicada a la expansión de variables de entorno, la cual recibe todos los parámetros de la llamada. Simplemente debemos encontrar cuáles de esos parámetros debemos expandir y reemplazar su valor con el de la variable.\\
		\lstinputlisting[linerange={96-104}, numbers=none]{parsing.c}
		
	\newpage
	\section{Apéndice}
	En esta sección se mostrará el código completo de los archivo modificados del esqueleto. Los archivos estarán ordenados por orden alfabético.\\
	
	\lstinputlisting{exec.h}
	\bigskip\bigskip\bigskip
	\lstinputlisting{exec.c}
	\bigskip\bigskip\bigskip
	\lstinputlisting{parsing.c}
	
	
			
\end{document}
