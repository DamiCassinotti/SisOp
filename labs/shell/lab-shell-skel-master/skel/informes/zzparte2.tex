\documentclass[a4paper, 12pt]{article}
\usepackage[T1]{fontenc}
\usepackage[scale=1,angle=0,opacity=1,color=black!60]{background}
\usepackage{tikzpagenodes}
\usepackage{lastpage}
\usepackage{lmodern}
\usepackage{float}
\usepackage[textwidth=420pt,textheight=630pt]{geometry}
\setlength{\oddsidemargin}{15.5pt}
%\usepackage[none]{hyphenat} %no cortar palabras

\usepackage[spanish, activeacute]{babel} %Definir idioma español
\usepackage[utf8]{inputenc} %Codificacion utf-8
\backgroundsetup{contents={}} %Saca el 'draft'
\definecolor{mygray}{rgb}{0.9,0.9,0.9}

\usepackage{listings}
\lstset{
    basicstyle=\footnotesize,
    backgroundcolor=\color{mygray},         
    breaklines=true,
    breakatwhitespace=true,   
    postbreak=\mbox{\textcolor{red}{$\hookrightarrow$}\space},              
    captionpos=b,                    
    keepspaces=true,                 
    numbers=left,                    
    numbersep=5pt,                  
    showspaces=false,                
    showstringspaces=false,
    showtabs=false,
    tabsize=4,
    language=C,
    frame=none,
    title=\lstname
}

\def\labelitemi{$\bullet$}

\begin{document}		
	% TÍTULO, AUTORES Y FECHA
	\begin{titlepage}
		\vspace*{\fill}
		\begin{center}
			\Large 75.08 Sistemas Operativos \\
			\Huge Lab Shell - Parte 2 \\
			\bigskip\bigskip\bigskip
			\large\textbf{Nombre y Apellido:} Damián Cassinotti \\
			\textbf{Padrón:} 96618 \\
			\textbf{Fecha de Entrega:} 04/05/2018\\
					
		\end{center}
		\vspace*{\fill}
	\end{titlepage}
	\pagenumbering{arabic}
	\newpage
			
	% ÍNDICE
	\tableofcontents
	\newpage
	%\pagenumbering{arabic}
	
	\section{Invocación avanzada}
		En esta segunda parte del lab veremos opciones avanzadas en la invicación de comandos. Para eso nos basaremos en el esqueleto provisto y la resolución de la parte anterior.\\
		
		Para esta parte se codificarán soluciones para la utilización de comandos built-in, la adición de variables de entorno y la ejecución de procesos en segundo plano.\\
		
		En cada caso veremos una breve explicación de lo pedido junto con el código de la solución. No mostraremos el código del shell completo, sino en cada parte nos centraremos en el necesario para cumplir con lo pedido. Se aclara que los ejercicios son iterativos e incrementales. Es decir, en cada ejercicio se suponen hechos los anteriores, contando con el código necesario.\\
		
		Al finalizar las explicaciones de cada punto se incluirá el código completo de los archivos modificados, sin incluir el esqueleto entero.
		\subsection{Comandos built-in}
		El primer objetivo del lab consiste en la ejecución de comandos built-in dentro de nuestra shell. Estos comandos son aquellos que no se pueden ejecutar en un proceso separado pues carecerían de sentido. En nuestro ejemplo se programarán los comandos \textit{cd, exit y pwd}.\\
		
		A continuación se presenta el código que se agregó o modificó del esqueleto del shell. Recordar que en caso de archivos ya existentes, se mostrará solamente la sección modificada.\\
		\lstinputlisting[linerange={7-7,18-31,62-62}, numbers=none]{../runcmd.c}
		\bigskip\bigskip\bigskip
		\lstinputlisting{../builtin.c}
		\bigskip\bigskip\bigskip
		\lstinputlisting[linerange={109-122}, numbers=none]{../exec.c}
		
		\textbf{¿Entre cd y pwd, alguno de los dos se podría implementar sin necesidad de ser built-in? ¿por qué? ¿cuál es el motivo, entonces, de hacerlo como built-in?} El comando pwd podría implementarse en un proceso nuevo. El motivo de hacerlo como built-in es que al realizar un fork no solo se crea un nuevo proceso, sino que se realizan distintas acciones como copia de memoria, etc. Para evitar todo esto, ya que pwd se trata solo de imprimir por pantalla la ruta actual, se decide hacerlo built-in para que sea más eficiente. \\
		Por otro lado, cd es obligatorio hacerlo como built-in ya que sino no tendría sentido, pues cambiaría el working directory del proceso hijo (que luego de esto, moriría) y no del proceso padre (o sea, la shell).
		
		\subsection{Variables de entorno adicionales}
		Este objetivo del lab consta de incluir variables de entorno adicionales, específicas para el proceso hijo. Para esto, luego del fork, se setean cada una de las variables de entorno que se reciben en el comando mediante la invocación a la función setenv.\\
		
		A continuación se muestra el código de la solución:\\
		\lstinputlisting[linerange={32-40, 58-61}, numbers=none]{../exec.c}
		
		\textbf{¿por qué es necesario hacer las llamadas a setenv(3) luego de la llamada a fork(2)?} Esto es necesario ya que las nuevas variables de entorno se deben agregar solamente en el proceso hijo, pues solo deben existir para la ejecución del proceso que se está llamando. En caso de hacerlo antes del fork, las nuevas variables estarán seteadas para el proceso padre (la shell) y perdurarán en el tiempo.\\
		
		|textbf{Supongamos, que en vez de utilizar setenv(3) por cada una de las variables, se guardan en un array y se lo coloca en el tercer argumento de una de las funciones de exec(3). ¿El comportamiento es el mismo que en el primer caso? Explicar qué sucede y por qué.} El comportamiento no será el mismo. Las funciones de la familia exec(3) que permiten el pasaje de variables de entorno tomarán como válidas sólo las variables que se encuentren en el array pasado. Es decir, si en la llamada se le pasan dos variables de entorno que quisimos agregar, esas serán todas las variables disonibles en la ejecución. Análogamente, si el array se encuentra vació, no existinrán variables de entorno válidas.\\
		Para emular el comportamiento deseado, se le debe pasar el array de cadenas que encontramos en la variable \textit{environ} que se encuentra definida en \textit{unistd.h}. Dicha variable posee todas las variables de entorno disponibles. Entonces, para poder agregar variables de entorno mediante la función exec(3) debemos obtener el array de todas las variables, agregarle las adicionales que queramos, y recién ahí lo podremos pasar como parámetro.
	
		\subsection{Procesos en segundo plano}
		El último de los objetivos de este lab es realizar una llamada a procesos que se ejecuten en segundo plano. \\
		
		Para esta resolución se tuvo en cuenta, luego de hacer el fork, qué tipo de ejecucón se estaba realizando. En todos los casos se crea un proceso hijo sobre el cual se ejecutará el comando deseado. La diferencia radica en el proceso padre: si la ejecución es por background, no se espera que el proceso hijo termine de realizar su ejecución. Caso contrario, se ejecuta la función \textit{waitpid}.\\
		
		A continuación veremos el código que se utilizó para esta solución: \\
		\lstinputlisting[linerange={7-7,52-55,62-62}, numbers=none]{../runcmd.c}
		\bigskip\bigskip\bigskip
		\lstinputlisting[linerange={72-72,63-65,80-83,107-107}, numbers=none]{../exec.c}
		
	\newpage
	\section{Apéndice}
	En esta sección se mostrará el código completo de los archivos modificados del esqueleto. Los archivos estarán ordenados por orden alfabético.\\
	
	\lstinputlisting{../builtin.h}
	\bigskip\bigskip\bigskip
	\lstinputlisting{../exec.c}
	\bigskip\bigskip\bigskip
	\lstinputlisting{../runcmd.c}
	
	
			
\end{document}
