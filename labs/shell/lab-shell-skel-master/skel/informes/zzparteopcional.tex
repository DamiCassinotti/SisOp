\documentclass[a4paper, 12pt]{article}
\usepackage[T1]{fontenc}
\usepackage[scale=1,angle=0,opacity=1,color=black!60]{background}
\usepackage{tikzpagenodes}
\usepackage{lastpage}
\usepackage{lmodern}
\usepackage{float}
\usepackage[textwidth=420pt,textheight=630pt]{geometry}
\setlength{\oddsidemargin}{15.5pt}
%\usepackage[none]{hyphenat} %no cortar palabras

\usepackage[spanish, activeacute]{babel} %Definir idioma español
\usepackage[utf8]{inputenc} %Codificacion utf-8
\backgroundsetup{contents={}} %Saca el 'draft'
\definecolor{mygray}{rgb}{0.9,0.9,0.9}

\usepackage{listings}
\lstset{
    basicstyle=\footnotesize,
    backgroundcolor=\color{mygray},         
    breaklines=true,
    breakatwhitespace=true,   
    postbreak=\mbox{\textcolor{red}{$\hookrightarrow$}\space},              
    captionpos=b,                    
    keepspaces=true,                 
    numbers=left,                    
    numbersep=5pt,                  
    showspaces=false,                
    showstringspaces=false,
    showtabs=false,
    tabsize=4,
    language=C,
    frame=none,
    title=\lstname
}

\def\labelitemi{$\bullet$}

\begin{document}		
	% TÍTULO, AUTORES Y FECHA
	\begin{titlepage}
		\vspace*{\fill}
		\begin{center}
			\Large 75.08 Sistemas Operativos \\
			\Huge Lab Shell - Challenges promocionales \\
			\bigskip\bigskip\bigskip
			\large\textbf{Nombre y Apellido:} Damián Cassinotti \\
			\textbf{Padrón:} 96618 \\
			\textbf{Fecha de Entrega:} 29/06/2018\\
					
		\end{center}
		\vspace*{\fill}
	\end{titlepage}
	\pagenumbering{arabic}
	\newpage
			
	% ÍNDICE
	\tableofcontents
	\newpage
	%\pagenumbering{arabic}
	
	\section{Challenges promocionales}
		\subsection{Pseudo-variables}
		El objetivo de este ejercicio es implementar la psudo-variable \$?. El propósito de esta variable es expandirse al estado de salida del último proceso ejecutado en primer plano.\\
		Otras dos variables mágicas pueden ser:
		\begin{itemize}
			\item \$ (\$\$): Esta variable mágica se expande al pid del shell. En un subshell se expande al pid del shell padre, no al del subshell.
			\item 0 (\$0): Se expande al nombre del shell. Esta variable se setea al inicializar el shell. 
		\end{itemize}
		
		Para la resolución del ejercicio se utilizó la variable global status, y se modificó la función de expansión de variables:\
		\lstinputlisting[linerange={9-9}, numbers=none]{../parsing.h}
		\bigskip\bigskip
		\lstinputlisting[linerange={106-124}, numbers=none]{../parsing.c}
		
		\subsection{Tuberías múltiples}
		El diseño que se eligió para la realización de tuberías múltiples es una forma recursiva de armar los comandos. En ele squeleto del shell se dividía el comando por el caracter "|", y, en el caso de utilizarse pipes, se trataban los dos lados del pipe como comandos simples.
		De esta forma, al usar tuberías múltiples, el comando de la derecha es el que tendría los distintos pipes. Entonces lo que se decidió hacer es tratar al comando de la derecha como una linea nueva, de forma que no importe cuantos pipes existan, se generarán tantos comandos como sean necesarios.
		\lstinputlisting[linerange={201-214}, numbers=none]{../parsing.c}
		
		\subsection{Segundo plano avanzado}
		
	
		
	\newpage
	\section{Apéndice}
	En esta sección se mostrará el código completo de los archivos modificados del esqueleto. Los archivos estarán ordenados por orden alfabético.\\
	
	\lstinputlisting{../parsing.h}
	\bigskip\bigskip\bigskip
	\lstinputlisting{../parsing.c}
	
	
			
\end{document}
