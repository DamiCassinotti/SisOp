\documentclass[a4paper, 12pt]{article}
\usepackage[T1]{fontenc}
\usepackage[scale=1,angle=0,opacity=1,color=black!60]{background}
\usepackage{tikzpagenodes}
\usepackage{lastpage}
\usepackage{lmodern}
\usepackage{float}
\usepackage[textwidth=420pt,textheight=630pt]{geometry}
\setlength{\oddsidemargin}{15.5pt}
\usepackage[none]{hyphenat} %no cortar palabras

\usepackage[spanish, activeacute]{babel} %Definir idioma español
\usepackage[utf8]{inputenc} %Codificacion utf-8
\backgroundsetup{contents={}} %Saca el 'draft'

\usepackage{listings}
\lstset{
    basicstyle=\footnotesize,
    breakatwhitespace=false,         
    breaklines=true,                 
    captionpos=b,                    
    keepspaces=true,                 
    numbers=left,                    
    numbersep=5pt,                  
    showspaces=false,                
    showstringspaces=false,
    showtabs=false,                  
    tabsize=4,
    language=C,
    frame=single,
    title=\lstname 
}

\def\labelitemi{$\bullet$}

\begin{document}		
	% TÍTULO, AUTORES Y FECHA
	\begin{titlepage}
		\vspace*{\fill}
		\begin{center}
			\Large 75.08 Sistemas Operativos \\
			\Huge Lab Unix - Parte 1 \\
			\bigskip\bigskip\bigskip
			\large\textbf{Nombre y Apellido:} Damián Cassinotti \\
			\textbf{Padrón:} 96618 \\
			\textbf{Fecha de Entrega:} 06/04/2018\\
					
		\end{center}
		\vspace*{\fill}
	\end{titlepage}
	\pagenumbering{arabic}
	\newpage
			
	% ÍNDICE
	\tableofcontents
	\newpage
	%\pagenumbering{arabic}
	
	\section{Comandos obligatorios}
		En esta primera parte del Lab se implementarán versiones simplificadas de comandos típicos de Unix. Por cada comando pedido se indicará un resumen de lo pedido y luego el código fuente utilizado para la solución.
		\subsection{rm0}
		Para el comando rm se pidió una implementación en la cual se puedan eliminar archivos regulares. La precondición que se debe cumplir es que el archivo que se pasará por parámetro existe y es regular.\\
		A continuacón se mostrará el código de los archivos utilizados.
		\lstinputlisting{rm0.h}
		\bigskip\bigskip\bigskip
		\lstinputlisting{rm0.c}
		
		\subsection{cat0}
		Para nuestra implementación del comando cat se debe mostrar por pantalla el contenido de un archivo. A diferencia de la implementación nativa, solo mostraremos un solo archivo regular (el cual, por precondicion, existe y se tienen permisos de lectura).\\
		La solución propuesta es la siguiente: 
		\lstinputlisting{cat0.h}
		\bigskip\bigskip\bigskip
		\lstinputlisting{cat0.c}
	
		\subsection{touch0}
		En este caso, cat0 implementará la función más utilizada del comando touch: crear un archivo vacío. No se tendrá en cuenta la modificación de fechas para archivos existentes. Es decir, si el archivo no existe se crea, y si existe no se hace nada.\\
		El código fuente de la solución es el siguiente:
		\lstinputlisting{touch0.h}
		\bigskip\bigskip\bigskip
		\lstinputlisting{touch0.c}
		
		\subsection{stat0}
		El comando stat muestra información de un archivo. A diferencia de esto, el comando stat0 será una versión reducida la cual solo mostrará el tamaño (en bytes), el nombre y el tipo de archivo (regular o directorio). Como precondición se indica que el archivo existe.\\
		La solución al problema es la siguiente:	
		\lstinputlisting{stat0.h}
		\bigskip\bigskip\bigskip
		\lstinputlisting{stat0.c}
		
	\newpage
	\section{Comandos opcionales}
		\subsection{rm1}
		En este comando opcional se agrega una extensión al comando rm0 desarrollado anteriormente. Para este caso se elimina la precondición que indica que el archivo pasado por parámetro debe ser regular. De esta forma, se debe hacer un manejo de errores para el caso en que el archivo sea un directorio.\\
		Para logarlo se debe manejar la variable errno y la función perror. La oslución propuesta es la siguiente:
		\lstinputlisting{rm1.h}
		\bigskip\bigskip\bigskip
		\lstinputlisting{rm1.c}
			
\end{document}
