\documentclass[a4paper, 12pt]{article}
\usepackage[T1]{fontenc}
\usepackage[scale=1,angle=0,opacity=1,color=black!60]{background}
\usepackage{tikzpagenodes}
\usepackage{lastpage}
\usepackage{lmodern}
\usepackage{float}
\usepackage[textwidth=420pt,textheight=630pt]{geometry}
\setlength{\oddsidemargin}{15.5pt}
%\usepackage[none]{hyphenat} %no cortar palabras

\usepackage[spanish, activeacute]{babel} %Definir idioma español
\usepackage[utf8]{inputenc} %Codificacion utf-8
\backgroundsetup{contents={}} %Saca el 'draft'
\definecolor{mygray}{rgb}{0.95,0.95,0.95}

\usepackage{listings}
\lstset{
    basicstyle=\footnotesize,
    backgroundcolor=\color{mygray},         
    breaklines=true,
    breakatwhitespace=true,   
    postbreak=\mbox{\textcolor{red}{$\hookrightarrow$}\space},              
    captionpos=b,                    
    keepspaces=true,                 
    numbers=left,                    
    numbersep=5pt,                  
    showspaces=false,                
    showstringspaces=false,
    showtabs=false,
    tabsize=4,
    language=C,
    frame=none,
    title=\lstname
}

\def\labelitemi{$\bullet$}

\begin{document}		
	% TÍTULO, AUTORES Y FECHA
	\begin{titlepage}
		\vspace*{\fill}
		\begin{center}
			\Large 75.08 Sistemas Operativos \\
			\Huge Lab Unix - Parte 3 \\
			\bigskip\bigskip\bigskip
			\large\textbf{Nombre y Apellido:} Damián Cassinotti \\
			\textbf{Padrón:} 96618 \\
			\textbf{Fecha de Entrega:} 13/04/2018\\
					
		\end{center}
		\vspace*{\fill}
	\end{titlepage}
	\pagenumbering{arabic}
	\newpage
			
	% ÍNDICE
	\tableofcontents
	\newpage
	%\pagenumbering{arabic}
	
	\section{Comandos obligatorios}
		La tercera parte del lab de Unix consiste en el desarrollo de otros comandos: tee, ls, cp y, opcionalmente, ps. Dichos comandos serán versiones simplificadas de los originales. Por cada comando se dará un breve resumen de lo pedido, así como también se contemplarán las precondiciones tomadas.
		\subsection{tee0}
		El comando \textit{tee0} deberá tomar como parámetro un archivo, e imprimir tanto en dicho archivo como por salida estándar todo lo que se obtenga de la entrada estándar. Para esto se utilizaron las syscalls open, read, write y close. En caso de que el archivo no exista, se crea; caso contrario, se sobreescribe.\\\\
		A continuacón se mostrará el código de los archivos utilizados.
		\lstinputlisting{tee0.h}
		\bigskip\bigskip\bigskip
		\lstinputlisting{tee0.c}
		
		\subsection{ls0}
		El comando \textit{ls0} será una implementación equivalente a \textit{ls --format=single-column --sort=none}, es decir, se listarán todos los archivos del directorio de a uno por linea y sin un orden determinado. La solución consiste en recorrer cada entrada del directorio sobre el cual se está ejecutando el proceso.\\\\
		La solución propuesta es la siguiente: 
		\lstinputlisting{ls0.h}
		\bigskip\bigskip\bigskip
		\lstinputlisting{ls0.c}
	
		\subsection{cp1}
		El último de los comandos obligatorios a realizar es \textit{cp}. A diferencia del comando \textit{cp0} propuesto en la Parte 2 de este mismo lab, esta vez debemos utilizar las syscalls mmap y memcpy.\\\\
		Como mmap debe reservar memoria tanto para el archivo original como para la copia, se debe optimizar para estar a salvo de tener que copiar archivos muy grandes y no contar con la memoria necesarias. \\\\
		Para esto se tuvo en cuenta la restricción que tiene el offset de mmap. Este offset debe ser un múltiplo del tamaño de página. Entonces lo propuesto es lo siguiente: en caso de que el tamaño de lo que falte copiar sea mayor al tamaño de página, se copian tantos bytes como ocupe la página, seteando el offset correspondiente. Caso contrario se copia lo faltante. De esta forma nos aseguramos de que el offset siempre sea un múltiplo entero del tamaño de página. \\\\
		El código fuente de la solución es el siguiente:
		\lstinputlisting{cp1.h}
		\bigskip\bigskip\bigskip
		\lstinputlisting{cp1.c}
		
	\newpage
	\section{Comandos opcionales}
		\subsection{ps0}
		El comando opcional a desarrollas en esta parte del lab es \textit{ps}. En este caso se pide mostrar información de los procesos que se encuentran corriendo. Esta información es el identificado del proceso (pid) y el nombre del programa usado para lanzar dicho proceso.\\\\
		En esta primera aproximación utiiizamos la información provista por el archivo comm dentro del directorio \textit{/proc/[pid]}. Dicho archivo contiene el nombre del programa buscado. De esta forma, ya contamos con toda la información necesaria para la resolución del problema. \\\\
		El código fuente de la solución es el siguiente:
		\lstinputlisting{ps0.h}
		\bigskip\bigskip\bigskip
		\lstinputlisting{ps0.c}
		
		\subsection{ps1}
		Una segunda versión del comando \textit{ps} se propone en la sección \textit{challenge del challenge}. En esta nueva versión, llamada ps1, se mostrará más información acerca de los procesos. \\\\
		La información a mostrar debe ser obtenida del archivo stat dentro del directorio del proceso mediante \textit{scanf}. La información elegida para mostrar es el pid, el programa que creó el proceso, el estado del proceso, el id del proceso padre (ppid), el grupo del proceso y la sesión sobre el cual fue creado.\\\\
		Esta versión del comando es equivalente a ejecutar \textit{ps -eo pid,comm,state,ppid,pgrp,session}.\\\\
		El código fuente de la solución es el siguiente:
		\lstinputlisting{ps1.h}
		\bigskip\bigskip\bigskip
		\lstinputlisting{ps1.c}
		
			
\end{document}
