\documentclass[a4paper, 12pt]{article}
\usepackage[T1]{fontenc}
\usepackage[scale=1,angle=0,opacity=1,color=black!60]{background}
\usepackage{tikzpagenodes}
\usepackage{lastpage}
\usepackage{lmodern}
\usepackage{float}
\usepackage[textwidth=420pt,textheight=630pt]{geometry}
\setlength{\oddsidemargin}{15.5pt}
\usepackage[none]{hyphenat} %no cortar palabras

\usepackage[spanish, activeacute]{babel} %Definir idioma español
\usepackage[utf8]{inputenc} %Codificacion utf-8
\backgroundsetup{contents={}} %Saca el 'draft'

\usepackage{listings}
\lstset{
    basicstyle=\footnotesize,
    breakatwhitespace=false,         
    breaklines=true,                 
    captionpos=b,                    
    keepspaces=true,                 
    numbers=left,                    
    numbersep=5pt,                  
    showspaces=false,                
    showstringspaces=false,
    showtabs=false,                  
    tabsize=4,
    language=C,
    frame=single,
    title=\lstname 
}

\def\labelitemi{$\bullet$}

\begin{document}		
	% TÍTULO, AUTORES Y FECHA
	\begin{titlepage}
		\vspace*{\fill}
		\begin{center}
			\Large 75.08 Sistemas Operativos \\
			\Huge Lab Unix - Parte 2 \\
			\bigskip\bigskip\bigskip
			\large\textbf{Nombre y Apellido:} Damián Cassinotti \\
			\textbf{Padrón:} 96618 \\
			\textbf{Fecha de Entrega:} 06/04/2018\\
					
		\end{center}
		\vspace*{\fill}
	\end{titlepage}
	\pagenumbering{arabic}
	\newpage
			
	% ÍNDICE
	\tableofcontents
	\newpage
	%\pagenumbering{arabic}
	
	\section{Comandos obligatorios}
		En esta segunda parte del Lab se implementarán más versiones simplificadas de comandos típicos de Unix. Por cada comando pedido se indicará un resumen de lo pedido y luego el código fuente utilizado para la solución. En caso de haber preguntas en el enunciado, las mismas serán respondidas luego del código.
		\subsection{ln0}
		Para el comando ln se pidió una implementación en la cual se puedan crear soft links a archivos. \\
		A continuacón se mostrará el código de los archivos utilizados.
		\lstinputlisting{ln0.h}
		\bigskip\bigskip\bigskip
		\lstinputlisting{ln0.c}
		
		\textbf{¿Qué ocurre si se intenta crear un enlace a un archivo que no existe?]} En caso de querer crear un soft link sobre un archivo inexistente se creará un dangling link (enlace colgante). Dicho enlace apuntará "hacia la nada", pues su archivo destino no existe. Si lo intentamos abrir desde la terminal mediante el comando cat, el mismo dará un error indicando que el archivo no existe.  
		
		\subsection{mv0}
		En el caso del comando mv, se pidió una implementación que mueva archivos regulares o directorios a otros. Para esto se debe renombrar el enlace a dicho archivo (regular o directorio). \\
		A continuacón se mostrará el código de los archivos utilizados.
		\lstinputlisting{mv0.h}
		\bigskip\bigskip\bigskip
		\lstinputlisting{mv0.c}
		
		\textbf{¿Se puede usar mv0 para renombrar archivos dentro del mismo directorio?} Sí, se puede usar mv para renombrar archivos. Como se usa el rename, se pueden renombrar el archivo manteniendo el path intacto.
		
		\subsection{cp0}
		Nuestra implementación del comando cp sólo tendrá en cuenta la copia de arhcivos regulares. Si el archivo destino existe se reemplaza; caso contrario, el archivo se crea. \\
		A continuacón se mostrará el código de los archivos utilizados.
		\lstinputlisting{cp0.h}
		\bigskip\bigskip\bigskip
		\lstinputlisting{cp0.c}
		
	\newpage
	\section{Comandos opcionales}
		\subsection{touch1}
		En este comando opcional se agrega una extensión al comando touch0 desarrollado en la parte 1 de este lab. En esta nueva implementación, además de crear archivos en caso de no existir, se deberán modificar las fechas de acceso y modificiación de un archivo existente, alterando su metadata.\\
		A continuacón se mostrará el código de los archivos utilizados.
		\lstinputlisting{touch1.h}
		\bigskip\bigskip\bigskip
		\lstinputlisting{touch1.c}
		
	\subsection{ln1}
		Esta versión de ln contemplará el caso de la creación de hard links (recordar que ln0 solamente creaba soft links).\\
		A continuacón se mostrará el código de los archivos utilizados.
		\lstinputlisting{ln1.h}
		\bigskip\bigskip\bigskip
		\lstinputlisting{ln1.c}
		
		\textbf{¿Cuál es la diferencia entre un hard link y un soft link?} La diferencia entre un soft link y un hard link radica principalmente en la forma de acceder a los contenidos.\\
		En la caso del soft link, éstos se interpretan en tiempo de ejecución como si el contenido del enlace fuese sustituido por el contenido del path al cual apunta. Además el soft link, tal como se vio anteriormente, puede referir a un archivo inexistente. \\
		Por otror lado, el hard link se usa exactamente igual que el archivo original para cualquier operación. Ambos refieren al mismo archivo (físico) y son indistinguibles, es decir que no se puede saber cuál es el archivo original.
		\\ 
		\\		
		\textbf{Crear un hard link a un archivo, luego eliminar el archivo original ¿Qué pasa con el enlace? ¿Se perdieron los datos del archivo?} Al tratarse de un hard link, ambos archivos apuntan directamente al mismo espacio físico de memoria. Por esto, al eliminar el archivo original lo único que se hace es eliminar una de las referencias a dicha memoria.\\ 
		Ya que los archivos son indistinguibles, podremos borrar uno o el otro sin perder los datos. Se podría decir que, en el fondo, todo archivo es un hard link.
		\\ 
		\\	
		\textbf{Repetir lo mismo, pero con un soft link. ¿Qué pasa ahora con el enlace? ¿Se perdieron los datos esta vez?} En este caso, al eliminar el archivo origen sí se pierden los datos, creandose un dangling link.\\
		Esto se debe a que, a diferencia de los hard links, los soft links apuntan hacia un path. Si dicho path no existe, no habrá forma de volver a recorrer el camino necesario en memoria para acceder a los datos.
			
\end{document}
